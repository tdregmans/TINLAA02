%%%%%%%%%%%%%%%%%%%%%%%%%%%%%%%%%%%%%%%%%%%%%%%%%%%%
%%%%%%% Verslag Tinlab Advanced Algoritms %%%%%%%
%%%%%%%%%%%%%%%%%%%%%%%%%%%%%%%%%%%%%%%%%%%%%%%%%%%%

\documentclass{article}
\usepackage[font=small,labelfont=bf]{caption}
\usepackage[dutch]{babel}

\usepackage[a4paper, total={6in, 8in}]{geometry}

%Includes "References" in the table of contents
\usepackage[nottoc]{tocbibind}
\usepackage{imakeidx} % package om indices te maken
\usepackage{amsmath} % package voor formules
\usepackage{graphicx} % package voor het invoegen van images
\usepackage{listings} % package voor het invoegen van listings (bron code)
\usepackage[colorlinks=true, allcolors=blue]{hyperref} % package voor hyperlinks (intern als extern)
\usepackage{xcolor} % package voor standaard kleuren
\usepackage{parskip} % package voor het formateren van paragraven
\usepackage{multirow} % package voor formateren tabellen --- \multicolumn{x}{|c|}{tekst}
\usepackage{ulem} % package om zinnen door te strepen met \sout{}
\usepackage[many]{tcolorbox} % for COLORED BOXES (tikz and xcolor included)

% biliotheek packages
\usepackage[backend=biber, style=alphabetic, sorting=nty]{biblatex} % package voor bibliotheek (bronnen)
\usepackage{csquotes} % ondersteund de biblatex package voor optimale weergave
\addbibresource{references.bib} % voeg eigen bibliotheek toe

% definieer kleuren: {naam}{kleurstijl}{waardes}
\definecolor{codegreen}{rgb}{0,0.6,0}
\definecolor{codegray}{rgb}{0.5,0.5,0.5}
\definecolor{codepurple}{rgb}{0.58,0,0.82}
\definecolor{backcolour}{rgb}{0.95,0.95,0.92}

% definieer stijl van lijsten
\lstdefinestyle{mystyle}{
    backgroundcolor=\color{backcolour},   
    commentstyle=\color{codegreen},
    keywordstyle=\color{magenta},
    numberstyle=\tiny\color{codegray},
    stringstyle=\color{codepurple},
    basicstyle=\ttfamily\footnotesize,
    breakatwhitespace=false,         
    breaklines=true,                 
    captionpos=b,                    
    keepspaces=true,                 
    numbers=left,                    
    numbersep=5pt,                  
    showspaces=false,                
    showstringspaces=false,
    showtabs=false,                  
    tabsize=2
}
\lstset{style=mystyle} % verander standaard stijl naar eigen stijl

% stijl voor box
\newtcolorbox{boxA}{
    fontupper = \bf,
    boxrule = 1.5pt,
    colframe = black % frame color
}

% stijl voor hyperlinks
\hypersetup{
    colorlinks=true,
    linkcolor=blue,
    filecolor=magenta,      
    urlcolor=cyan,
    }
\urlstyle{same}


\begin{document}

	\sffamily
	%%%%%%% Front page %%%%%%%
	%%%%%%%%%%%%%%%%%%%%%%%%%%%%%%%%%%%%%%%%%%%%%%%%%%%%
	
	\begin{titlepage}
	
		\centering
		  \vfill
		  {\bfseries\Huge
		    Groepsverslag Tinlab Advanced Algorithms \\
		      \vskip2cm
		    }
		    {\bfseries\Large
		      Thijs Dregmans, Tobias de Bildt, Eliam Traas, Hidde-Jan Daniëls\\
		    }
		    {
		      \bfseries\normalsize
		      1024272, 1023603, 1003233, 0943798\\
		      \vskip1cm
		      \today\\
		  }    
		  \vfill
		  \includegraphics[width=4cm]{logohr.png} % also works with logo.pdf
		  \vfill
		  \vfill
	    
	\end{titlepage}
	
	\newpage
	
	%%%%%%% Table of Content %%%%%%%
	%%%%%%%%%%%%%%%%%%%%%%%%%%%%%%%%%%%%%%%%%%%%%%%%%%%%
	
	\tableofcontents
	
	\newpage

	%%%%%%% Analyse %%%%%%%
	%%%%%%%%%%%%%%%%%%%%%%%%%%%%%%%%%%%%%%%%%%%%%%%%%%%%
	
	\section{Analyse}
	
		% Als je een sluis gaat modelleren is het uiteraard eerst zaak een goed beeld op te
		% bouwen van de werking van een sluis. Stel je, voordat je het internet overhoop
		% haalt, eerst eens wat relevante vragen:
		% - Zijn er verschillende soorten sluizen?
		% - Uit welke onderdelen bestaat een sluis?
		% - Hoe lang duurt het, voordat een boot door de sluis heen is?
		% - Wat voor stappen (states?) moeten doorlopen worden om een boot van
		%   de ene naar de andere kant te krijgen?
		% - Wat voor cijfers horen er bij het bovenstaande? Hoe lang duurt een
		%   bepaalde stap?
		% - etc

		[text]

		%%%%%%%%%%%%%%%%%%%%%%%%%%%%%%%%%%%%%%%%%%%%%%%%%%%%
	
	\newpage
	
	%%%%%%% Requirements %%%%%%%
	%%%%%%%%%%%%%%%%%%%%%%%%%%%%%%%%%%%%%%%%%%%%%%%%%%%%
	
	\section{Requirements}
	
		% Hoe zijn de wensen van de opdrachtgever ge¨ınterpreteerd? Tot wat voor
		% requirements/specificaties leiden deze? Anders gezegd: Wat betekent
		% veilig, effici¨ent, etc. en wat heb je aan bronnen geraadpleegd om tot een
		% goede analyse te komen? (Dit laatste hoef je niet te beschrijven: het
		% blijkt immers uit citaten of verwijzingen die je gebruikt.)

		[text]

		%%%%%%%%%%%%%%%%%%%%%%%%%%%%%%%%%%%%%%%%%%%%%%%%%%%%
	
	\newpage

	%%%%%%% Modeleren %%%%%%%
	%%%%%%%%%%%%%%%%%%%%%%%%%%%%%%%%%%%%%%%%%%%%%%%%%%%%
	
	\section{Modeleren}
	
		% -  De modelcriteria van Vaandrager zijn op allerlei manieren tegenstrij-
		%    dig. Welke keuzes en afwegingen heb je gemaakt en waarom?
		% –  Gemodelleerde onderdelen.
		% –  Werking van het model.

		[text]

		%%%%%%%%%%%%%%%%%%%%%%%%%%%%%%%%%%%%%%%%%%%%%%%%%%%%
	
	\newpage

	%%%%%%% Verificatie %%%%%%%
	%%%%%%%%%%%%%%%%%%%%%%%%%%%%%%%%%%%%%%%%%%%%%%%%%%%%
	
	\section{Verificatie}

		% – Wat heb je geverifieerd, waarom en hoe?
		% – Als je iets niet kon verifi¨eren, waarom dan niet?
		% – Een harde eis is dat er een aantal eigenschappen geverifieerd zijn.
		% We modelleren een systeem immers middels Kripke structuren om
		% harde uitspraken over eigenschappen van zo’ n systeem te kunnen
		% doen.

		[text]

		%%%%%%%%%%%%%%%%%%%%%%%%%%%%%%%%%%%%%%%%%%%%%%%%%%%%
	
	\newpage
	
	%%%%%%% references %%%%%%%
	%%%%%%%%%%%%%%%%%%%%%%%%%%%%%%%%%%%%%%%%%%%%%%%%%%%%
	
	\bibliographystyle{plain} 
	\bibliography{references} % Entries are in the `references.bib` file
	% Importing references while no references are made, will cause an error!!!
	
\end{document}
